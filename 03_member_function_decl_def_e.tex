\documentclass{article}

\begin{document}

\large
{\bf{Member function is declared and defined at the same time.}}
\normalsize

\begin{verbatim}
struct S {
  int f(){ A x; /* ... */ return a; }  // OK
  typedef int A
  int a;
};
\end{verbatim}
In this example,
before {\tt{typedef-name}} `{\tt{A}}' and member variable `{\tt{a}}'
are declared, member function `{\tt{f}}' refers to `{\tt{A}}' and
`{\tt{a}}'. To parse correctly, it's necessary to save member
function `{\tt{f}}' body once and after do parse.

\begin{verbatim}
struct S {
  int f()  /* { A x; ... return a; } are saved.
              Do like int f(); */
  int a;
} /* At this point, `S' becomes complete type, and do parse saved
     member function `f' body. */
;
\end{verbatim}
If it's possible to realize parsing saved member function body
by just calling `{\tt{yyparse}}'. We have to change the nubmer
of initial state:
\begin{verbatim}
state 100

  557 function_definition: function_definition_begin1 . function_body

    '{'  shift, and go to state 234

    function_body       go to state 235
    compound_statement  go to state 236
\end{verbatim}
In this case, change to `100'. Also note that tokens must be
saved one.
\begin{verbatim}
int yyparse()
{
  int yystate;
  ...
  yystate = 0; // Usual initial state number
  if (saved member function parse analysis)
    yystate = 100;  // overwrite!
}
\end{verbatim}
When non-terminal `{\tt{function\_definition}}' is reduced,
it's necessary to return from `{\tt{yyparse}}':
\begin{verbatim}
state 102

    6 declaration: function_definition .

    $default  reduce using rule 6 (declaration)
\end{verbatim}
In this case, change `yyparse' like bellow:
\begin{verbatim}
int yyparse()
{
  ...

  // Reduce code. Frome here.
  ...
  // Reduce code. To here.

  if (yystate == 102 && saved member function parse analysis)
    return 0;  // Parse analysis is done normally. So return 0.

  ...
} 
\end{verbatim}

\end{document}
